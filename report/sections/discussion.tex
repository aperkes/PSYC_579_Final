\section{Discussion}

Subjects performed best with green and red lights. While these differences may
appear too small to be meaningful, at $35$ mph, a car travels $51$ feet per
second. A difference of $.1$s translates to over $5$ ft traveled, which could
be extremely significant for whatever is in those $5$ ft.  Further analysis of
the data shows that there was also a meaningful difference in the spread of
reaction times. While the means were quite similar, different colors had a
higher proportion of slow-responding outliers. The misses (marked in X's in
Figure \ref{fig:results}.) also represent dramatic differences in visibility,
as some lights tended to disappear into the background of similarly colored
distractors.  There are a range of factors that may influence why different
colors performed differently. We must acknowledge the possibility that our
backgrounds (which were varied, but by no means randomly created or chosen)
might favor certain colors in a way that is not representative of driving
experience. Additionally, this is by no means a perfectly controlled situation,
and a whole range of confounding factors could be influencing the results.  One
question we had initially was whether red was a proper choice for bike tail
lights. By convention, bike tail lights match cars' red lights. It seemed
possible then that the low signal to noise ration could make red more difficult
to spot. However, this was not the case. It seems likely that in driving
environments, we are particularly attentive to lights of certain colors (green
and red lights have particular meaning, and red lights are the steady signal
for the objects we tend to look for). This could certainly compensate for the
effects of noise. Similarly, pink, white, blue, and cyan lights rarely have
relevance when driving. Obviously, this is speculation, but it is reassuring to
know that our tail light conventions aren't necessarily making us less safe.
One of the initial motivations for this study was to understand what factors
make bikers more visible. In researching prior work, we came across a study
showing that reflective clothing, especially on the legs, improves visibility
far more than bikers predict \cite{wood2013bicyclists}. In preparing the
experiment, we found that the cyclist's body was a far more salient stimulus
(popping out so strongly that it requiring significant modification to make the
comparatively small light source a factor.) Thus cyclists interested in
visibility should perhaps worry less about tail light color, and instead add
reflective clothing to make their body more visible.

