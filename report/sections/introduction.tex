\section{Introduction}

Visual search is a task in which the subject in an experiment is asked to
identify a target item amongst a sea of distractor items. Researchers designing
visual search experiments, such as ourselves, are usually interested in the time
it takes for the subject to find the target. It is not hard to imagine that
tasks end up falling into one of two categories: ``pop-out'' and ``serial
search''. In the first case, the target is readily apparent, and the response
time of the subject is independent of the number of distractors. In the latter
case, the target is better hidden, and the subject must examine each item in a
sequence. As a result, reaction time increases with the number of distractors,
and the increase per distractor is double per ``no'' responses than for ``yes''
responses. 


In our experiment, we were primarily interested in the amount of time it takes
for a subject to identifying a cyclist through varying amounts of visual
``clutter''. Furthermore, we were interested in seeing if and how different
colored tail lights on a cyclist would affect the time it takes for a subject
to identify them. 
