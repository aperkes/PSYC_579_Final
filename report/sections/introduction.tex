\section{Introduction}

%% Visual search is a task in which the subject in an experiment is asked to
%% identify a target item amongst a sea of distractor items. Researchers designing
%% visual search experiments, such as ourselves, are usually interested in the time
%% it takes for the subject to find the target. It is not hard to imagine that
%% tasks end up falling into one of two categories: ``pop-out'' and ``serial
%% search''. In the first case, the target is readily apparent, and the response
%% time of the subject is independent of the number of distractors. In the latter
%% case, the target is better hidden, and the subject must examine each item in a
%% sequence. As a result, reaction time increases with the number of distractors,
%% and the increase per distractor is double per ``no'' responses than for ``yes''
%% responses. 
%% 
%% 
%% In our experiment, we were primarily interested in the amount of time it takes
%% for a subject to identifying a cyclist through varying amounts of visual
%% ``clutter''. Furthermore, we were interested in seeing if and how different
%% colored tail lights on a cyclist would affect the time it takes for a subject
%% to identify them. 


An estimated 48,000 cyclists were injured in motor vehicle traffic accidents in
2013, over half of which occurred in low light conditions [NHTSA, 2015].
Drivers often fail to observe cyclists, making cyclists particularly vulnerable
[Wood et al, 2013]. Visual Search studies provide insight into why this occurs.
Visibility is certainly an issue: bike lights are often quite dim in comparison
to car lights, and the intensity of the signal predictably influences
performance [Engel, 1977]. Attention also dramatically alters perception. Since
bikers represent rare events, drivers are not attentive to their presence.
Additionally, visual search is influenced dramatically by the azimuth [Treisman
\& Gelade, 1980]. A bike in a bike lane falls somewhere between
$30^\circ-90^\circ$ lateral of the usual center of vision of a driver.
Visual acuity drops dramatically as eccentricity increases, making cyclists
potentially invisible to drivers. [Wolfe et al, 1998] Here we seek to
determine how color affects biker recognition in a pseudo-realistic
context. Because of the variety of factors involved, we developed an
experimental procedure to test visual search for bikers against a backdrop
of natural scenes filled with common distractors: cars, billboards, street
lamps, and stop lights.
